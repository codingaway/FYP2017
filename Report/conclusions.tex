\chapter{Conclusions}
\label{conclusions}
This project was set out to implement a parallel GA using MPI and profiling the performance issue over a sequential version by varying the number of parallel processes and population size. Initially, we expected to see a substantial performance gain by parallelising the SGA. By implementing the Parallel GA many underlying issues and techniques were discovered that were unknown to me previously. 

In MPI the communication cost is major concern. If a program is not carefully designed to minimise the communication overhead the overall performance can suffer a lot. As we have discovered, there are three very different models of parallel GA. Out of these three only the first model(Master-slave) implemented in this project. The experiments done in this project shown that the traditional master-slave model of PGA using MPI does not result in much performance gain over the sequential SGA because of the communication cost involved for message-passing. In general MPI programs are more suitable for computationally heavy program. If the computational cost is not greater than the communication cost then its performance gain will be negative. The subsequent versions of the PGA (v3), the GA behaviour was changed due to the fact that it is very close to \textit{Island} model. This model of parallel GA needs a migration policy implemented. The time constraint was a major factor for this challenging project. If time permitted, the next step would be to implement a distributed model of Parallel GA with a migration policy.


This project was an exploratory experience in using MPI library and a Cluster computing environment. In this project I have had a great experience in exploring many difficult and exciting concepts in Parallelisation, High Performance Computing(HPC) and some aspect of Genetic Algorithms(GA). I have learned the advantages and limitation of MPI libraries and many underlying challenging issues involved. I gained a lot of experience in setting up a Beowulf cluster, configuring and administration of Linux Server, using remote development environment and shell scripting. Last but not least learned to use \LaTeX\  to produce this report. 